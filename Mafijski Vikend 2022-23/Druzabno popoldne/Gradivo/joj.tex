\documentclass[a4paper]{article}

\usepackage[utf8]{inputenc}
\usepackage[T1]{fontenc}
\usepackage[slovene]{babel}
\usepackage{amssymb}
\usepackage{amsmath}  % razna okolja za poravnane enačbe ipd.
\usepackage{amsthm}   % definicije okolij za izreke, definicije, ...
\usepackage{xypic}
\usepackage{graphicx} % za vstavljanje slik
\usepackage{booktabs} % za lepše tabele
\usepackage{multirow} % za vnose v tabeli čez več vrstic
\usepackage{siunitx}
\begin{document}
\begin{enumerate}
    \item\begin{enumerate}
    \item    krog  (3)
    \item krožnica (7)
    \item premica (1)
    \item kolobar (5)
    \end{enumerate}
    \item\begin{enumerate}
        \item  $2023$  (9)  
        \item $0$  (2)
        \item $1$ (6)
        \item $2$ (0)
    \end{enumerate}
    \item \begin{enumerate}
        \item      sode (8)
            \item lihe (4)
            \item nobenega (3)
            \item vse (1)
            \end{enumerate}
          \item  \begin{enumerate}
                \item $2.047.277$  (9)    
                \item $2024\cdot x$ (6)
                \item $2023!+1$ (8)
                \item $x$ (2)
                \end{enumerate}
\end{enumerate}
\newpage
Zapiši rešitev enačbe
\[x^{x^4}=64\]
v obliki \(a^{\frac{b}{c}}\), kjer je koda \(abc\).\\
\\
\\



Koda so prve 3 števke večje izmed rešitev enačbe
\(3^x + 9^x=27^x\).\\
\\
\\


Koliko je končna temperatura vode v $K$, če zmešamo $2l$ vode s $30^{\circ}C$ in $4l$ vode z $0^{\circ}C$.\\
\\
\\


Izračunaj vsoto vrste na 2 decimalki natančno:
$$\sum_{n=1}^{\infty}\arctan(n+1)-\arctan(n).$$\\
\\
\\



Vlaka, ki sta na začetni razdalji $100~km$, se vozita eden proti drugemu vsak s hitrostjo $50~km/h$. 
Med njima leti muha s hitrostjo $100~km/h$, 
izmenično od enega do drugega. Kolikšno pot opravi muha med letom?
\newpage
5949


234



283


114\\
\\






Kje je zadnja pozicija kralja pred koncem igre?
\begin{enumerate}
    \item h7 $\implies 2.05$
   \item h8 $\implies 1.05$
   \item  g7 $\implies 0.05$
   \item g8 $\implies 5.05$
\end{enumerate}
\newpage

    \[2a^2+1=b\cdot a^2\]

\end{document}