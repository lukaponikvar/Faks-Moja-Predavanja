\documentclass[sumniki]{izpit}

\usepackage{multicol}
\usepackage{enumerate}

\newcommand{\Rat}{\mathbb{Q}}
\newcommand{\Rea}{\mathbb{R}}
\newcommand{\Nat}{\mathbb{N}}
\newcommand{\Com}{\mathbb{C}}
\def\R{\small \noindent{\bf Rešitev: } \normalsize}
\newcommand{\dl}{\ \mathrm d}

\setboolean{@celostranske}{true}


\begin{document}

\izpit[naloge = 4]
  {Kolokvij}{21.~januar~2023}{
%  Čas pisanja je 90 minut.
  Možno je doseči 100 točk. Vse odgovore dobro utemelji! Veliko uspeha!
}


\naloga[\tocke{25}] Poišči vsa kompleksna števila \(a \in \mathbb{C}\), pri katerih ima naslednji sistem enačb natanko eno rešitev:
\[|z-4|=2|z-1|\]
\[|z-a|+|z-2a|=|a|\]
Dobljeno množico ustreznih vrednosti za \(a\) grafično ponazori.
\naloga[\tocke{25}] Ugotovi, za katere \(n \in \mathbb{N}\) obstaja limita
\[\lim_{x \to \infty}\frac{2023\left(\sqrt[n]{1+x^3}-1\right)-x^3}{(e^{x^2}-1-x^2)\sin^2 x}.\]
Kadar obstaja, jo izračunaj.
\naloga[\tocke{25}] Naj bo \(n \geq 2\) in \(\{v_1,\ldots,v_n\}\) baza realnega vektorskega prostora \(V\). 
Utemelji za katera naravna števila \(n\) je množica 
\[\{v_1+v_2,v_2+v_3,\ldots,v_n+v_1\}\]
baza prostora \(V\).

\naloga[\tocke{25}] Izračunaj vsoto v odvisnosti od $x \in \mathbb{R}$
\[1+ \sum_{n=1}^{2023}\left(n\cdot\sin^2(nx)\right)+\sum_{i=1}^{2023}\left(i\cdot\cos^2(ix)\right).\]
\end{document}
