\documentclass{amsart}
\usepackage[T1]{fontenc}
\usepackage[utf8]{inputenc}
\usepackage[slovene]{babel}
\usepackage{textcase}
\usepackage{amsmath,amsfonts,amssymb,amsthm}
\usepackage{xypic}
\usepackage{booktabs}
\usepackage{multirow}
\usepackage{siunitx}
\sisetup{output-decimal-marker={,},group-separator={.}}


\newcommand{\RR}{\mathbb{R}}
\newcommand{\NN}{\mathbb{N}}
\newcommand{\PP}{\mathbb{P}}
\newcommand{\ZZ}{\mathbb{Z}}
\newcommand{\geslo}[2]{\noindent\textbf{#1} \quad \hangindent=1cm #2\\[-1pc]}



{\theoremstyle{theorem}
\newtheorem{izrek}{Izrek}[section]
\newtheorem{lema}[izrek]{Lema}
\newtheorem{trditev}[izrek]{Trditev}
\newtheorem{posledica}[izrek]{Posledica}
}
{\theoremstyle{definition}
\newtheorem{definicija}[izrek]{Definicija}
\newtheorem{vaja}[izrek]{Vaja}
\newtheorem{prob}[izrek]{Nemogoč problem}
\newtheorem{opomba}[izrek]{Opomba}
\newtheorem{domneva}[izrek]{Domneva}
\newtheorem{odprt}[izrek]{Odprt problem}
\newtheorem{primer}[izrek]{Primer}
}
\def\qed{$\hfill\Box$}   % konec dokaza
\def\qedm{\qquad\Box}   % konec dokaza v matematičnem načinu


% \title[kratek naslov]{poln naslov}
\title[Prosta Boolova algebra]{Prosta Boolova algebra\\\Large Seminar}
% Podatki o avtorjih
\author{Luka Ponikvar}
\address{Luka Ponikvar\\
Fakulteta za matematiko in fiziko\\
Oddelek za matematiko\\
Jadranska 21\\
1000 Ljubljana\\
Slovenija}
\email{lp29353@student.uni-lj.si}

\begin{document}

\maketitle


\begin{abstract}
    Pričetek toerije Boolovih algeber je nepresenetljivo pripisan Geogu Boolu,
    ki 
\end{abstract}


%%%%%%%%%%%%%%%%%%%%%%%%%%%%%%%%%%%%%%%%%%%%%%%%%%%%%%%%%%%%%%%%%%%%%%%%%%%%%%%%%%%%%%%%%%%%%%%%%%%%%%%%%%%%%%%%%%%%%%%%%%%%%%%%
\section{Boolove algebre}

\begin{definicija}[Boolova algebra]
    {\bf Boolova algebra} je neprazna množica \(A\), skupaj z binarnima operacijama \(\vee\) \footnote{Imenujemo jo ``join''\ oz.\ 
    ``ali''.} in \(\wedge\)\footnote{Imenujemo jo ``meet''\ oz.\ ``in''.}, unarno operacijo \(\neg\)
    \footnote{Imenujemo jo negacija, označujemo pa tudi kot \('\).} in dvema odlikovanima 
    elementoma \(0\) in \(1\), ki skupaj zadoščajo sledečim aksiomom\footnote{Negacija ima najvišjo prioriteto, medtem ko ima ``in'' višjo prioriteto kot ``ali''.}:

    \begin{align}
        \label{eq0}
        \neg 0 &= 1, & \neg 1 =& 0,\\ \label{eq1}
        p \wedge 0 &= 0, & p \vee 1 =& 1,\\ \label{eq2}
        p \wedge 1 &= p, & p \vee 0 =& p,\\ \label{eq3}
        p \wedge \neg p &= 0, & p \vee \neg p =& 1,\\ \label{eq4}
         \neg (\neg p) &= p,\\ \label{eq5}
        p \wedge p &= p, & p \vee p =& p,\\ \label{eq6}
        \neg (p \wedge q) &= \neg p \vee \neg q, & \neg (p \vee q) =& \neg p \wedge \neg q,\\ \label{eq7}
        p \wedge q &= q \wedge p, & p \vee q =& q \vee p,\\ \label{eq8}
        p \wedge (q \wedge r) &= (p \wedge q) \wedge r, & p \wedge (q \vee r) =& (p \vee q) \vee r,\\ \label{eq9}
        p \wedge (q \vee r) &= (p \wedge q) \vee (p \wedge r), & p \vee (q \wedge r) =& (p \vee q) \wedge (p \vee r).
    \end{align}
    
\end{definicija}

\begin{primer}[Izrojena Boolova algebra]
    Najenostavnejši je primer izrojene Boolove algebre, ki je potenčna množica prazne množice:
    \begin{align*}
        \mathcal{P}\left(\emptyset\right) = \{\emptyset\}.
    \end{align*}
    Operacije na tej množici definiramo kot konstantne preslikave, ki vse slikajo v \[0 = 1 = \emptyset.\]
\end{primer}

\begin{primer}[Boolova algebra z dvema elementoma]
    Najmanjši primer neizrojene Boolove algebre je potenčna množica enojca\footnote{Element množice smo označili kar z \(\infty\).}:
    \begin{align*}
        \mathcal{P}\left(\{\infty\}\right) = \{\emptyset, \{\infty\}\}.
    \end{align*}
    Taka Boolova algebra ima le dva elementa:
    \begin{align*}
        \emptyset = 0, && \{\infty\} = 1.
    \end{align*}
    Operaciji join in meet sta predstavljeni z naslednjima tabelama:
    \begin{align*}
    \begin{tabular}{c|c|c|}
        \(\vee\) & $0$ & $1$ \\
        \hline
        $0$ & \(0\) &$1$ \\
        \hline
        $1$ & \(1\) & $1$ \\
        \hline
    \end{tabular}&& \text{in} &&
    \begin{tabular}{c|c|c|}
        \(\wedge\) & $0$ & $1$ \\
        \hline
        $0$ & $0$ &$0$ \\
        \hline
        $1$ & $0$ & $1$ \\
        \hline
    \end{tabular},
\end{align*}
komplementacija, pa $0$ preslika v $1$ in obratno.
\end{primer}

\begin{primer}[Končno-končna Boolova algebra]
    Najpreprostejši primer Boolove algebre je potenčna množica neprazne množice \(X\), ki jo seveda opremimo z operacijami unije,
    preseka in komplementa. 
    
    Malce splošnejši primer je, da si ogledamo določeno podmnožico \(\mathcal{P}\left(X\right)\).
    Če definiramo \(A := \{B \subset X~|~B~\text{končna  ali } B'~\text{končna}\}\), tudi dobimo Boolovo algebro, imenovano
    končno-končna Boolova Algebra.

    Lahko se tudi ne omejimo le na končne, ampak na števne množice in dobimo števno-števno Boolovo algebro.
    Premislek deluje za poljubno kardinalnost, je pa to težje dokazati.
\end{primer}

%%%%%%%%%%%%%%%%%%%%%%%%%%%%%%%%%%%%%%%%%%%%%%%%%%%%%%%%%%%%%%%%%%%%%%%%%%%%%%%%%%%%%%%%%%%%%%%%%%%%%%%%%%%%%%%%%%%%%%%%%%%%%%%%
\section{Princip dualnosti}

\begin{definicija}[Boolov polinom]
    {\bf Boolov polinom} je izraz, sestavljen iz konstant \(0\) in \(1\), neznank \(p_0\), \(\ldots\) , \(p_n\), s pomočjo standardnih operacij
    meet, join in komplementa.
\end{definicija}

\begin{primer}[Boolov polinom]
    Primer polinoma je
    \[p \wedge (q \vee r).\]
    Ta polinom, pa je na pogled zelo podoben polinomu
    \[p \vee (q \wedge r),\]
    kar motivira naslednjo definicijo.
\end{primer}

\begin{definicija}[Dualnost]
    Naj bo \(f(p_1, \ldots ,p_n)\) Boolov polinom v n spremenljivkah. Takemu polinomu lahko priredimo
    tri nove polinome:
    \begin{enumerate}
        \item {\bf komplement polinoma} \(f(p_1, \ldots ,p_n)\):
            \[f'(p_1, \ldots ,p_n),\]
        \item {\bf dual polinoma} \(f(p_1, \ldots ,p_n)\):
            \[f'(p_1', \ldots ,p_n'),\]
        \item {\bf kontradual polinoma} \(f(p_1, \ldots ,p_n)\):
            \[f(p_1', \ldots ,p'_n).\]
    \end{enumerate}
\end{definicija}
\begin{opomba}
    V resnici obstaja grupa \(G\), ki deluje na množici \(\mathcal{BP}\) vseh Boolovih polinomov.
    Obstajajo štiri funkcije, ki bijektivno preslikajo množico \(\mathcal{BP}\) nazaj nase:
    \begin{enumerate}
        \item Identična funkcija:
        \[id: \mathcal{BP} \rightarrow \mathcal{BP}\]
        \[id: f(p_1, \ldots ,p_n) \mapsto f(p_1, \ldots ,p_n).\]
        \item Komplementna funkcija:
        \[c: \mathcal{BP} \rightarrow \mathcal{BP}\]
        \[c: f(p_1, \ldots ,p_n) \mapsto f'(p_1, \ldots ,p_n).\]
        \item Dualna funkcija:
        \[d: \mathcal{BP} \rightarrow \mathcal{BP}\]
        \[d: f(p_1, \ldots ,p_n) \mapsto f'(p_1', \ldots ,p_n').\]
        \item Kontradualna funkcija:
        \[k: \mathcal{BP} \rightarrow \mathcal{BP}\]
        \[k: f(p_1, \ldots ,p_n) \mapsto f(p_1', \ldots ,p_n').\]
    \end{enumerate}

    Grupa \(G\) je zaprta za operacijo \(\circ\):
    \begin{table*}[htp]
        \begin{tabular}{c|c|c|c|c|}
            \(\circ\) & $id$ & $c$ & $d$ & $k$ \\
            \hline
            $id$ & $id$ &$c$ & $d$ & $k$ \\
            \hline
            $c$ & $c$ & $id$ & $k$ & $d$ \\
            \hline
            $d$ & $d$ & $k$ & $id$ & $c$ \\
            \hline
            $k$ & $k$ & $d$ & $c$ & $id$ \\
            \hline
        \end{tabular}.
    \end{table*}\\
    Opazimo, da je \(G \cong \ZZ_2  \oplus \ZZ_2 \), torej je Kleinova četverka.
\end{opomba}

\begin{opomba}
    Praktična posledica principa dualnosti je moč dokazati le polovico izrekov in trditev, 
    saj druga polovica sledi iz tega principa.
\end{opomba}

%%%%%%%%%%%%%%%%%%%%%%%%%%%%%%%%%%%%%%%%%%%%%%%%%%%%%%%%%%%%%%%%%%%%%%%%%%%%%%%%%%%%%%%%%%%%%%%%%%%%%%%%%%%%%%%%%%%%%%%%%%%%%%%%
\section{Urejenost}

V tem razdelku delujemo v poljubni Boolovi algebri A.

\begin{lema}
    \label{lema1}
    \(\left(p \vee q\right) \wedge p = p\) in \(\left(p \wedge q\right) \vee p = p\).
\end{lema}

\begin{proof} 
    \begin{align*}
        \left(p \vee q\right) \wedge p &\stackrel{\ref{eq2}}{=} \left(p \vee q\right) \wedge \left(p \vee 0 \right) \\
                    &\stackrel{\ref{eq9}}{=} p \vee \left(q \wedge 0\right)\\
                    &\stackrel{\ref{eq1}}{=} p \vee 0\\
                    &\stackrel{\ref{eq2}}{=} p 
    \end{align*}
    Druga furmula sledi iz dualnosti.

\end{proof}


\begin{lema}
    \label{lema2}
    \(p \wedge q = p\) natanko tedaj ko \(p \vee q = q\).
\end{lema}

\begin{proof}
    Če je \(p \wedge q = p\), je 
    \begin{align*}
        p \vee q &\stackrel{}{=} \left(p \wedge q\right) \vee q \\
                    &\stackrel{\ref{eq9}}{=} \left(p \vee q\right) \wedge \left(q \vee q\right) \\
                    &\stackrel{\ref{eq5}}{=} \left(p \vee q\right) \wedge q\\
                    &\stackrel{\ref{lema1}}{=} q 
    \end{align*}
    Drugo implikacijo dobimo z zamenjavo p in q, ter formiranjem dualov.
\end{proof}

\begin{definicija}
    Na vsaki Boolovi algebri lahko vpeljemo urejenost kot:
    \begin{center}
        \(p \leq q \) natanko tedaj ko \(p \wedge q  = p.\)
    \end{center}
\end{definicija}

\begin{lema}
    Relacija \(\leq\) je delna urejenost.
\end{lema}

\begin{proof}
    Refleksivnost sledi iz \ref{eq5}, antisimetričnost pa sledi iz \ref{eq7}: če je \(p \leq q\) in \(q \leq p\), potem
    je \(p = p \wedge q = q \wedge p = q\). Tranzitivnost sledi iz \ref{eq8}:
    če je \(p \leq q\) in \(q \leq r\) je \(p\wedge r = \left(p \wedge q\right) \wedge r = p \wedge \left(q \wedge r\right) = p \wedge q = p\)
\end{proof}

\begin{lema}
    \label{lema3}
    \begin{enumerate}
        \item \(0 \leq p\) in \(p \leq 1\).
        \item Če \(p \leq q\) in \(r \leq s\), potem \(p \wedge r \leq q \wedge s\) in \(p \vee r \leq q \vee s\). \label{lema3.2}
        \item Če \(p \leq q\), potem \(q' \leq p'\).
    \end{enumerate}
\end{lema}

\begin{proof}
    Prva točka je očitna. Druga sledi iz definicije in \ref{lema2}. Tretja točka sledi s komplemetiranjem.

\end{proof}

\begin{definicija}[Meje]
    Če je $E$ podmnožica delno urejene Boolove algebre \(A\), lahko govorimo o množici \(F\) vseh zgornjih mej za \(E\).
    Element \(q\) pripada množici \(F\), če za vsak \(p \in E\) velja \(p \leq q\). Če ima \(F\) najmanjši element, je ta 
    enolično določen in ga imenujemo {\bf supremum} množice \(E\) oz. njena {\bf najmanjša zgornja meja}\footnote{Pišemo tudi natančna zgornja meja.}. Podobno 
    definiramo {\bf infimum} oz. {\bf največjo spodnjo mejo}\footnote{Pišemo tudi natančna spodnja meja.} množice \(E\)
\end{definicija}

\begin{primer}[Prazna množica]
    \label{primer1}
    Če je \(E = \emptyset\), je vsak element na prazno zgornja meja te množice. Tedaj ima \(E\) supremum, in sicer kar element \(0\)
    (\ref{lema3} točka \ref{lema3.2}). 
    Podoben razmislek nas privede do zaključka, da je infimum množice \(E\) element \(1\).
\end{primer}

\begin{primer}[Enojec]
    \label{primer2}
    Če je \(E = \{p\}\), je p hkrati zgornja in spodnja meja za \(E\). Sledi da je p tudi infimum in supremum.
\end{primer}

\begin{lema}
    Za vsaka \(p\) in \(q\) ima množica \(\{p, q\}\) za supremum element \(p \vee q\) in za infimum element \(\{p \wedge q\}\)
\end{lema}

\begin{proof}
    Očitno je \(p \vee q\) zgornja meja te množice. Zaradi točke \ref{lema3.2} v \ref{lema3}, pa je to tudi natančna zgornja meja:
    če je $p \leq r$ in $q \leq r$ je $p \wedge q \leq r \wedge r = r$.
    Drug del sledi iz dualnosti.

\end{proof}

\begin{opomba}[Posplošitev]
    Lemo bi lahko posplošili na poljubno končno neprazno množico \(E\). Za infimum pišemo \(\bigwedge E\), za supremum pa
    \(\bigvee E.\) Enake oznake uporabljamo za supremume in infimume poljubnih množic (če jih te seveda imajo).

    Primera \ref{primer1} in \ref{primer2} bi lahko sedaj zapisali kot:
    \begin{align*}
        \bigvee \emptyset = 0, && \bigwedge \emptyset = 1, && \bigvee \{p\} = p, && \bigwedge \{p\} = p.
    \end{align*}

    Če imamo opravka z množico \(\{p_i|~i \in I\}\), kjer je $I$ poljubna indeksna množica, pišemo tudi:
    \begin{align*}
        \bigvee_{i \in I} p_i && \bigwedge_{i \in I} p_i
    \end{align*}
\end{opomba}

%%%%%%%%%%%%%%%%%%%%%%%%%%%%%%%%%%%%%%%%%%%%%%%%%%%%%%%%%%%%%%%%%%%%%%%%%%%%%%%%%%%%%%%%%%%%%%%%%%%%%%%%%%%%%%%%%%%%%%%%%%%%%%%%
\section{Kompletne Boolove algebre}

Končno-končna Boolova algebra nad \(\NN\) je primer Boolove algebre, kjer nimajo vse podmnožice elementov
natančnih spodnjih oz. zgornjih mej. Primer take množice je množica vseh enojcev sodih naravnih števil. To motivira naslednjo definicijo.

\begin{definicija}[Kompletna Boolova algebra]
    Boolova algebra z lastnostjo, da ima vsaka njena podmnožica infimum in supremum se imenuje {\bf Kompletna
    Boolova Algebra}.
\end{definicija}

\begin{lema}
    Če je \(\{p_i\}\) družina elementov Boolove algebre, potem:
    \begin{align*}
        \left(\bigvee_{i} p_i\right)' = \bigwedge_{i} p_i' &&\text{in} && \left(\bigwedge_{i} p_i\right)' = \bigvee_{i} p_i'.
    \end{align*}
    Enačbi povesta, da obstoj ene strani implicira obstoj druge in njuno enakost.
\end{lema}

\begin{proof}
    
\end{proof}

\begin{posledica}[Zadosten pogoj za kompletnost]
    Če ima vsaka podmnožica Boolove algebre infimum (supremum), potem je ta Boolova algebra kompletna.
\end{posledica}

\begin{proof}
    
\end{proof}

Zdaj nas zanimajo še lastnosti natančnih zgornjih (spodnjih) mej. Natančneje njihova asociativnost, komutativnost
in distributivnost.

\begin{lema}[Asociativnost in komutativnost]
    Če je \(\{I_j\}\) družina množic z unijo \(I\), in če je \(p_i\) element Boolove algebre za vsak \(i \in I\), tedaj je
    \begin{align*}
        \bigvee_{j}\left(\bigvee_{i \in I_j} p_i\right) = \bigvee_{i \in I} p_i && \text{in} && \bigwedge_{j}\left(\bigwedge_{i \in I_j} p_i\right) = \bigwedge_{i \in I} p_i.
    \end{align*}
\end{lema}

\begin{proof}
    
\end{proof}

\begin{lema}
    
\end{lema}

\begin{proof}
    
\end{proof}

























\newpage
\section*{Angleško-slovenski slovar strokovnih izrazov}


\geslo{Boolean algebra}{Boolova algebra}

\geslo{Degenerate}{Izrojena}

\geslo{The principle of duality}{Princip dualnosti}

\geslo{Lower (upper) bound}{Spodnja(zgornja) meja}

\geslo{Least upper bound}{Najmanjša zgornja meja}

\geslo{Greatest lower bound}{Največja spodnja meja}

\geslo{Complete Boolean algebra}{Kompletna Boolova algebra}

\geslo{Finite-cofinite Boolean algebra}{Končno-končna Boolova algebra}

\geslo{Countable-cocountable Boolean algebra}{Števno-števna Boolova algebra}

\geslo{}{}

\geslo{}{}

\geslo{}{}

\geslo{}{}

\geslo{}{}

\geslo{}{}

\geslo{}{}

\geslo{}{}

\geslo{}{}

\geslo{}{}


\geslo{}{}






\begin{thebibliography}{1}
\bibitem{AiZ}
Givant, Steven; Halmos, Paul. “IntroDuction to Boolean Algebras (Undergraduate Texts in Mathematics),” Springer (2009).\bibitem{CaW}
N.~Calkin in H.~S.~Wilf, Recounting the rationals,
\emph{Amer.~Math.~Monthly}  \textbf{107}  (2000),  360--363.
\bibitem{Gra}
J.~Grasselli, \emph{Elementarna teorija števil}, DMFA -- založništvo, Ljubljana, 2009.
\end{thebibliography}

\end{document}
