\documentclass{amsart}
\usepackage[T1]{fontenc}
\usepackage[utf8]{inputenc}
\usepackage[slovene]{babel}
\usepackage{textcase}
\usepackage{amsmath,amsfonts,amssymb,amsthm}
\usepackage{xypic}
\usepackage{booktabs}
\usepackage{multirow}
\usepackage{siunitx}
\usepackage{bbm}
\sisetup{output-decimal-marker={,},group-separator={.}}


\newcommand{\RR}{\mathbb{R}}
\newcommand{\NN}{\mathbb{N}}
\newcommand{\PP}{\mathbb{P}}
\newcommand{\ZZ}{\mathbb{Z}}
\newcommand{\geslo}[2]{\noindent\textbf{#1} \quad \hangindent=1cm #2\\[-1pc]}



{\theoremstyle{theorem}
\newtheorem{izrek}{Izrek}[section]
\newtheorem{lema}[izrek]{Lema}
\newtheorem{trditev}[izrek]{Trditev}
\newtheorem{posledica}[izrek]{Posledica}
}
{\theoremstyle{definition}
\newtheorem{definicija}[izrek]{Definicija}
\newtheorem{vaja}[izrek]{Vaja}
\newtheorem{prob}[izrek]{Nemogoč problem}
\newtheorem{opomba}[izrek]{Opomba}
\newtheorem{domneva}[izrek]{Domneva}
\newtheorem{odprt}[izrek]{Odprt problem}
\newtheorem{primer}[izrek]{Primer}
}
\def\qed{$\hfill\Box$}   % konec dokaza
\def\qedm{\qquad\Box}   % konec dokaza v matematičnem načinu


% \title[kratek naslov]{poln naslov}
\title[Prosta Boolova algebra]{Prosta Boolova algebra\\\Large Seminar}
% Podatki o avtorjih
\author{Luka Ponikvar}
\address{Luka Ponikvar\\
Fakulteta za matematiko in fiziko\\
Oddelek za matematiko\\
Jadranska 21\\
1000 Ljubljana\\
Slovenija}
\email{lp29353@student.uni-lj.si}

\begin{document}

\maketitle


\begin{abstract}
    Pričetek toerije Boolovih algeber je nepresenetljivo pripisan Geogu Boolu,
    ki 
\end{abstract}


%%%%%%%%%%%%%%%%%%%%%%%%%%%%%%%%%%%%%%%%%%%%%%%%%%%%%%%%%%%%%%%%%%%%%%%%%%%%%%%%%%%%%%%%%%%%%%%%%%%%%%%%%%%%%%%%%%%%%%%%%%%%%%%%
\section{Boolove algebre}

\begin{definicija}[Boolova algebra]
    {\bf Boolova algebra} je neprazna množica \(A\), skupaj z binarnima operacijama \(\vee\) \footnote{Imenujemo jo ``join''\ oz.\ 
    ``ali''.} in \(\wedge\)\footnote{Imenujemo jo ``meet''\ oz.\ ``in''.}, unarno operacijo \(\neg\)
    \footnote{Imenujemo jo negacija, označujemo pa tudi kot \('\).} in dvema odlikovanima 
    elementoma \(0\) in \(1\), ki skupaj zadoščajo sledečim aksiomom\footnote{Negacija ima najvišjo prioriteto, medtem ko ima ``in'' višjo prioriteto kot ``ali''.}:

    \begin{align}
        \label{eq0}
        \neg 0 &= 1, & \neg 1 =& 0,\\ \label{eq1}
        p \wedge 0 &= 0, & p \vee 1 =& 1,\\ \label{eq2}
        p \wedge 1 &= p, & p \vee 0 =& p,\\ \label{eq3}
        p \wedge \neg p &= 0, & p \vee \neg p =& 1,\\ \label{eq4}
         \neg (\neg p) &= p,\\ \label{eq5}
        p \wedge p &= p, & p \vee p =& p,\\ \label{eq6}
        \neg (p \wedge q) &= \neg p \vee \neg q, & \neg (p \vee q) =& \neg p \wedge \neg q,\\ \label{eq7}
        p \wedge q &= q \wedge p, & p \vee q =& q \vee p,\\ \label{eq8}
        p \wedge (q \wedge r) &= (p \wedge q) \wedge r, & p \wedge (q \vee r) =& (p \vee q) \vee r,\\ \label{eq9}
        p \wedge (q \vee r) &= (p \wedge q) \vee (p \wedge r), & p \vee (q \wedge r) =& (p \vee q) \wedge (p \vee r).
    \end{align}
    
\end{definicija}

\begin{primer}[Izrojena Boolova algebra]
    Najenostavnejši je primer izrojene Boolove algebre, ki je potenčna množica prazne množice:
    \begin{align*}
        \mathcal{P}\left(\emptyset\right) = \{\emptyset\}.
    \end{align*}
    Operacije na tej množici definiramo kot konstantne preslikave, ki vse slikajo v \[0 = 1 = \emptyset.\]
\end{primer}

\begin{primer}[Boolova algebra z dvema elementoma]
    Najmanjši primer neizrojene Boolove algebre je potenčna množica enojca\footnote{Element množice smo označili kar z \(\infty\).}:
    \begin{align*}
        \mathbbm{2} = \mathcal{P}\left(\{\infty\}\right) = \{\emptyset, \{\infty\}\}.
    \end{align*}
    Taka Boolova algebra ima le dva elementa:
    \begin{align*}
        \emptyset = 0, && \{\infty\} = 1.
    \end{align*}
    Operaciji join in meet sta predstavljeni z naslednjima tabelama:
    \begin{align*}
    \begin{tabular}{c|c|c|}
        \(\vee\) & $0$ & $1$ \\
        \hline
        $0$ & \(0\) &$1$ \\
        \hline
        $1$ & \(1\) & $1$ \\
        \hline
    \end{tabular}&& \text{in} &&
    \begin{tabular}{c|c|c|}
        \(\wedge\) & $0$ & $1$ \\
        \hline
        $0$ & $0$ &$0$ \\
        \hline
        $1$ & $0$ & $1$ \\
        \hline
    \end{tabular},
\end{align*}
komplementacija, pa $0$ preslika v $1$ in obratno.
\end{primer}

\begin{primer}[Končno-končna Boolova algebra]
    Najpreprostejši primer Boolove algebre je potenčna množica neprazne množice \(X\), ki jo seveda opremimo z operacijami unije,
    preseka in komplementa. 
    
    Malce splošnejši primer je, da si ogledamo določeno podmnožico \(\mathcal{P}\left(X\right)\).
    Če definiramo \(A := \{B \subset X~|~B~\text{končna  ali } B'~\text{končna}\}\), tudi dobimo Boolovo algebro, imenovano
    končno-končna Boolova Algebra.

    Lahko se tudi ne omejimo le na končne, ampak na števne množice in dobimo števno-števno Boolovo algebro.
    Premislek deluje za poljubno kardinalnost, je pa to težje dokazati.
\end{primer}

%%%%%%%%%%%%%%%%%%%%%%%%%%%%%%%%%%%%%%%%%%%%%%%%%%%%%%%%%%%%%%%%%%%%%%%%%%%%%%%%%%%%%%%%%%%%%%%%%%%%%%%%%%%%%%%%%%%%%%%%%%%%%%%%
\section{Princip dualnosti}

\begin{definicija}[Boolov polinom]
    {\bf Boolov polinom} je izraz, sestavljen iz konstant \(0\) in \(1\), neznank \(p_0\), \(\ldots\) , \(p_n\), s pomočjo standardnih operacij
    meet, join in komplementa.
\end{definicija}

\begin{primer}[Boolov polinom]
    Primer polinoma je
    \[p \wedge (q \vee r).\]
    Ta polinom, pa je na pogled zelo podoben polinomu
    \[p \vee (q \wedge r),\]
    kar motivira naslednjo definicijo.
\end{primer}

\begin{definicija}[Dualnost]
    Naj bo \(f(p_1, \ldots ,p_n)\) Boolov polinom v n spremenljivkah. Takemu polinomu lahko priredimo
    tri nove polinome:
    \begin{enumerate}
        \item {\bf komplement polinoma} \(f(p_1, \ldots ,p_n)\):
            \[f'(p_1, \ldots ,p_n),\]
        \item {\bf dual polinoma} \(f(p_1, \ldots ,p_n)\):
            \[f'(p_1', \ldots ,p_n'),\]
        \item {\bf kontradual polinoma} \(f(p_1, \ldots ,p_n)\):
            \[f(p_1', \ldots ,p'_n).\]
    \end{enumerate}
\end{definicija}
\begin{opomba}
    V resnici obstaja grupa \(G\), ki deluje na množici \(\mathcal{BP}\) vseh Boolovih polinomov.
    Obstajajo štiri funkcije, ki bijektivno preslikajo množico \(\mathcal{BP}\) nazaj nase:
    \begin{enumerate}
        \item Identična funkcija:
        \[id: \mathcal{BP} \rightarrow \mathcal{BP}\]
        \[id: f(p_1, \ldots ,p_n) \mapsto f(p_1, \ldots ,p_n).\]
        \item Komplementna funkcija:
        \[c: \mathcal{BP} \rightarrow \mathcal{BP}\]
        \[c: f(p_1, \ldots ,p_n) \mapsto f'(p_1, \ldots ,p_n).\]
        \item Dualna funkcija:
        \[d: \mathcal{BP} \rightarrow \mathcal{BP}\]
        \[d: f(p_1, \ldots ,p_n) \mapsto f'(p_1', \ldots ,p_n').\]
        \item Kontradualna funkcija:
        \[k: \mathcal{BP} \rightarrow \mathcal{BP}\]
        \[k: f(p_1, \ldots ,p_n) \mapsto f(p_1', \ldots ,p_n').\]
    \end{enumerate}

    Grupa \(G\) je zaprta za operacijo \(\circ\):
    \begin{table*}[htp]
        \begin{tabular}{c|c|c|c|c|}
            \(\circ\) & $id$ & $c$ & $d$ & $k$ \\
            \hline
            $id$ & $id$ &$c$ & $d$ & $k$ \\
            \hline
            $c$ & $c$ & $id$ & $k$ & $d$ \\
            \hline
            $d$ & $d$ & $k$ & $id$ & $c$ \\
            \hline
            $k$ & $k$ & $d$ & $c$ & $id$ \\
            \hline
        \end{tabular}.
    \end{table*}\\
    Opazimo, da je \(G \cong \ZZ_2  \oplus \ZZ_2 \), torej je Kleinova četverka.
\end{opomba}

\begin{opomba}
    Praktična posledica principa dualnosti je moč dokazati le polovico izrekov in trditev, 
    saj druga polovica sledi iz tega principa.
\end{opomba}

%%%%%%%%%%%%%%%%%%%%%%%%%%%%%%%%%%%%%%%%%%%%%%%%%%%%%%%%%%%%%%%%%%%%%%%%%%%%%%%%%%%%%%%%%%%%%%%%%%%%%%%%%%%%%%%%%%%%%%%%%%%%%%%%
\section{Urejenost}

V tem razdelku delujemo v poljubni Boolovi algebri A.

\begin{lema}
    \label{lema1}
    \(\left(p \vee q\right) \wedge p = p\) in \(\left(p \wedge q\right) \vee p = p\).
\end{lema}

\begin{proof} 
    \begin{align*}
        \left(p \vee q\right) \wedge p &\stackrel{\eqref{eq2}}{=} \left(p \vee q\right) \wedge \left(p \vee 0 \right) \\
                    &\stackrel{\eqref{eq9}}{=} p \vee \left(q \wedge 0\right)\\
                    &\stackrel{\eqref{eq1}}{=} p \vee 0\\
                    &\stackrel{\eqref{eq2}}{=} p 
    \end{align*}
    Druga furmula sledi iz dualnosti.

\end{proof}


\begin{lema}
    \label{lema2}
    \(p \wedge q = p\) natanko tedaj ko \(p \vee q = q\).
\end{lema}

\begin{proof}
    Če je \(p \wedge q = p\), je 
    \begin{align*}
        p \vee q &\stackrel{}{=} \left(p \wedge q\right) \vee q \\
                    &\stackrel{\eqref{eq9}}{=} \left(p \vee q\right) \wedge \left(q \vee q\right) \\
                    &\stackrel{\eqref{eq5}}{=} \left(p \vee q\right) \wedge q\\
                    &\stackrel{\ref{lema1}}{=} q 
    \end{align*}
    Drugo implikacijo dobimo z zamenjavo p in q, ter formiranjem dualov.
\end{proof}

\begin{definicija}
    Na vsaki Boolovi algebri lahko vpeljemo urejenost kot:
    \begin{center}
        \(p \leq q \) natanko tedaj ko \(p \wedge q  = p.\)
    \end{center}
\end{definicija}

\begin{lema}
    Relacija \(\leq\) je delna urejenost.
\end{lema}

\begin{proof}
    Refleksivnost sledi iz \eqref{eq5}, antisimetričnost pa sledi iz \eqref{eq7}: če je \(p \leq q\) in \(q \leq p\), potem
    je \(p = p \wedge q = q \wedge p = q\). Tranzitivnost sledi iz \eqref{eq8}:
    če je \(p \leq q\) in \(q \leq r\) je \(p\wedge r = \left(p \wedge q\right) \wedge r = p \wedge \left(q \wedge r\right) = p \wedge q = p\)
\end{proof}

\begin{lema}
    \label{lema3}
    \begin{enumerate}
        \item \(0 \leq p\) in \(p \leq 1\).
        \item Če \(p \leq q\) in \(r \leq s\), potem \(p \wedge r \leq q \wedge s\) in \(p \vee r \leq q \vee s\). \label{lema3.2}
        \item Če \(p \leq q\), potem \(q' \leq p'\).
    \end{enumerate}
\end{lema}

\begin{proof}
    Prva točka je očitna. Druga sledi iz definicije in \ref{lema2}. Tretja točka sledi s komplemetiranjem.

\end{proof}

\begin{definicija}[Meje]
    Če je $E$ podmnožica delno urejene Boolove algebre \(A\), lahko govorimo o množici \(F\) vseh zgornjih mej za \(E\).
    Element \(q\) pripada množici \(F\), če za vsak \(p \in E\) velja \(p \leq q\). Če ima \(F\) najmanjši element, je ta 
    enolično določen in ga imenujemo {\bf supremum} množice \(E\) oz. njena {\bf najmanjša zgornja meja}\footnote{Pišemo tudi natančna zgornja meja.}. Podobno 
    definiramo {\bf infimum} oz. {\bf največjo spodnjo mejo}\footnote{Pišemo tudi natančna spodnja meja.} množice \(E\)
\end{definicija}

\begin{primer}[Prazna množica]
    \label{primer1}
    Če je \(E = \emptyset\), je vsak element na prazno zgornja meja te množice. Tedaj ima \(E\) supremum, in sicer kar element \(0\)
    (\ref{lema3} točka \ref{lema3.2}). 
    Podoben razmislek nas privede do zaključka, da je infimum množice \(E\) element \(1\).
\end{primer}

\begin{primer}[Enojec]
    \label{primer2}
    Če je \(E = \{p\}\), je p hkrati zgornja in spodnja meja za \(E\). Sledi da je p tudi infimum in supremum.
\end{primer}

\begin{lema}
    Za vsaka \(p\) in \(q\) ima množica \(\{p, q\}\) za supremum element \(p \vee q\) in za infimum element \(\{p \wedge q\}\)
\end{lema}

\begin{proof}
    Očitno je \(p \vee q\) zgornja meja te množice. Zaradi točke \ref{lema3.2} v \ref{lema3}, pa je to tudi natančna zgornja meja:
    če je $p \leq r$ in $q \leq r$ je $p \wedge q \leq r \wedge r = r$.
    Drug del sledi iz dualnosti.

\end{proof}

\begin{opomba}[Posplošitev]
    Lemo bi lahko posplošili na poljubno končno neprazno množico \(E\). Za infimum pišemo \(\bigwedge E\), za supremum pa
    \(\bigvee E.\) Enake oznake uporabljamo za supremume in infimume poljubnih množic (če jih te seveda imajo).

    Primera \ref{primer1} in \ref{primer2} bi lahko sedaj zapisali kot:
    \begin{align*}
        \bigvee \emptyset = 0, && \bigwedge \emptyset = 1, && \bigvee \{p\} = p, && \bigwedge \{p\} = p.
    \end{align*}

    Če imamo opravka z množico \(\{p_i|~i \in I\}\), kjer je $I$ poljubna indeksna množica, pišemo tudi:
    \begin{align*}
        \bigvee_{i \in I} p_i && \bigwedge_{i \in I} p_i
    \end{align*}
\end{opomba}

%%%%%%%%%%%%%%%%%%%%%%%%%%%%%%%%%%%%%%%%%%%%%%%%%%%%%%%%%%%%%%%%%%%%%%%%%%%%%%%%%%%%%%%%%%%%%%%%%%%%%%%%%%%%%%%%%%%%%%%%%%%%%%%%
\section{Kompletne Boolove algebre}

Končno-končna Boolova algebra nad \(\NN\) je primer Boolove algebre, kjer nimajo vse podmnožice elementov
natančnih spodnjih oz. zgornjih mej. Primer take množice je množica vseh enojcev sodih naravnih števil. To motivira naslednjo definicijo.

\begin{definicija}[Kompletna Boolova algebra]
    Boolova algebra z lastnostjo, da ima vsaka njena podmnožica infimum in supremum se imenuje {\bf kompletna
    Boolova algebra}.
\end{definicija}


\begin{opomba}
    Vsaka končna Boolova algebra je kompletna.
\end{opomba}

\begin{lema}
    Če je \(\{p_i\}\) družina elementov Boolove algebre, potem:
    \begin{align*}
        \left(\bigvee_{i} p_i\right)' = \bigwedge_{i} p_i' &&\text{in} && \left(\bigwedge_{i} p_i\right)' = \bigvee_{i} p_i'.
    \end{align*}
    Enačbi povesta, da obstoj ene strani implicira obstoj druge in njuno enakost.
\end{lema}

\begin{proof}
    Denimo, da je \(p = \bigvee_{i} p_i\). Ker je \(p_i \leq p\) za vsak $i$, sledi iz \ref{lema3}, da je \(p' \leq p_i'\) za vsak $i$.
    Če je \(q \leq p_i'\) za vse $i$, je \(p_i \leq q'\) za vse $i$ in tako po definiciji supremuma \(p \leq q'\) in \(q \leq p'\). 
    Torej je $p'$ res \(\bigwedge_{i} p_i'\).

    Dualni argument zadošča za dokaz v desno druge enačbe. Da dokažemo še obratno smer pa lahko dokazane lastnosti uporabimo na družinah komplementov.

\end{proof}

\begin{posledica}[Zadosten pogoj za kompletnost]
    Če ima vsaka podmnožica Boolove algebre infimum (supremum), potem je ta Boolova algebra kompletna.
\end{posledica}

% \begin{proof}
    
% \end{proof}

Zdaj nas zanimajo še lastnosti natančnih zgornjih (spodnjih) mej. Natančneje njihova asociativnost, komutativnost
in distributivnost. O komutativnosti ni smisla govoriti, saj je supremum (infimum) pripisan neki množici elementov, torej neodvisno od njihove urejenosti.

\begin{lema}[Asociativnost]
    Če je \(\{I_j\}\) družina množic z unijo \(I\), in če je \(p_i\) element Boolove algebre za vsak \(i \in I\), tedaj je
    \begin{align*}
        \bigvee_{j}\left(\bigvee_{i \in I_j} p_i\right) = \bigvee_{i \in I} p_i && \text{in} && \bigwedge_{j}\left(\bigwedge_{i \in I_j} p_i\right) = \bigwedge_{i \in I} p_i.
    \end{align*}
\end{lema}

\begin{proof}
    Označimo z \(q_j = \bigvee_{i \in I_j} p_i\) in s \(q = \bigvee_j q_j\).
    Za vsak \(i \in I\) obstaja \(j\), da je \(i \in I_j\). Torej za vsak \(i\)
    obstaja \(j\), da je \(p_i \leq q_j\), kar skupaj s \(q_j \leq q\) da \(p_i \leq q\) za vsak \(i\).
    Denimo sedaj, da obstaja tak \(r\), da je \(p_i \leq r\) za vse \(i\). Tedaj je še toliko bolj
    \(p_i \leq r\) za \(i \in I_j\) in po definiciji \(q_j \leq r\) za vsak \(j\).
    Torej je spet po definiciji \(q \leq r\), kar dokazuje, da je \(q\) res želeni supremum.

\end{proof}

\begin{lema}
    Če je \(p\) element in \(\{q_j\}\) družina elementov v neki Boolovi algebri, potem
    \begin{align*}
        p \wedge \bigvee_i q_i = \bigvee_i (p \wedge q_i).
    \end{align*}
\end{lema}

\begin{proof}
    Pišimo \(q = \bigvee_i q_i\). Ker velja \(p \wedge q_i \leq p \wedge q\) za vse \(i\),
    je torej element \(p \wedge q\) zgornja meja za \(\bigvee_i (p \wedge q_i).\)
    Denimo, da je tudi \(r\) zgornja meja. Tedaj velja
    \begin{align*}
        q_i \stackrel{\eqref{eq2}}{=} 1 \wedge q_i \stackrel{\eqref{eq3}}{=} (p \vee p') \wedge q_i \stackrel{\eqref{eq9}}{=} \left(p \wedge q_i\right) \vee
        \left(p' \wedge q_i\right) \leq r \vee \left(p' \wedge q_i\right) \stackrel{\ref{lema3}}{\leq}  r \vee p'.
    \end{align*}
    Po definiciji supremuma je \(q \leq r \vee p'\). Sledi
    \begin{align*}
        p \wedge q \leq p \wedge \left(r \vee p'\right) \stackrel{\eqref{eq9}}{=}
        \left(p \wedge r\right) \vee \left(p \wedge p'\right) \stackrel{\eqref{eq3}}{=} \left(p \wedge r\right) \vee 0 \stackrel{\eqref{eq2}}{=} \left(p \wedge r\right) \leq r.
    \end{align*}
\end{proof}

\begin{posledica}
    Če sta \(\{p_i\}\) in \(\{q_j\}\) družini elementov v Boolovi algebri, potem je 
    \begin{align*}
        \left(\bigvee_i p_i \right) \wedge \left(\bigvee_j q_j \right)=
        \bigvee_{i, j} \left(p_i \wedge q_j\right). 
    \end{align*}
\end{posledica}

\begin{definicija}[Kompletno distributivnostno pravilo]
    Naj bo \(A\) Boolova algebra, $I$ in $J$ pa taki indeksni množici, da za vsaka \(i \in I\) in \(j \in J\) 
    element \(p(i, j)\) leži v $A$. Pravimo, da družina \(\{p(i, j)\}\) zadošča {\bf kompletnemu distributivnostnemu pravilu}, če je 
    \begin{align}
        \label{eq11} \bigwedge_{i \in I} \bigvee_{j \in J} p(i, j) = \bigvee_{a \in J^{I}} \bigwedge_{i \in I} p(i, a(i)).
    \end{align}
\end{definicija}


% Ali mora biti Kompletna, ali se delamo da ne??
\begin{definicija}[Kompletno distributivna algebra]
    Boolova algebra A je {\bf kompletno distributivna}, ko ima naslednjo lastnost:
    ko vsi supremumi \(\bigvee_{j \in J} p(i, j)\) in infimumi \(\bigwedge_{i \in I} p(i, a(i))\) obstajajo za vsako družino
    \(\{p(i, j)\}\), potem obstoj leve strani \eqref{eq11} implicira obstoj desne strani in njuno enakost.
\end{definicija}



%%%%%%%%%%%%%%%%%%%%%%%%%%%%%%%%%%%%%%%%%%%%%%%%%%%%%%%%%%%%%%%%%%%%%%%%%%%%%%%%%%%%%%%%%%%%%%%%%%%%%%%%%%%%%%%%%%%%%%%%%%%%%%%%
\section{Podalgebre}

\begin{definicija}
    Boolova podalgebra Boolove algebre $A$ je neprazna podmnožica $B$ množice $A$, ki je z zožitvijo operacij Boolova algebra.
    Vsaka neizrojena Boolova Algebra $A$ ima trivialno podalgebro $\mathbbm{2}$, ostale imamo za netrivialne.\
    Vsaka Boolova algebra $A$ premore tudi nepravo podalgebro $A$, vse ostale podalgebre so prave.
\end{definicija}

\begin{opomba}
    Presek poljubne družine Boolovih podalgeber je ponovno podalgebra. Presek prazne družine porodi nepravo podalgebro.
\end{opomba}

Če vzamemo neko podmnožico \(E\) v Boolovi algebri \(A\), lahko tvorimo presek vseh podalgeber, ki vsebujejo $E$ (vsaj ena taka obstaja, namreč $A$).
Ta presek, recimo $B$ je najmanjša podalgebra, ki vsebuje $E$. Rečemo, da je $B$ generirana z $E$ oz.\ da je $E$ množica generatorjev za $B$.

\begin{primer}
    Vzemimo \(E = \emptyset\). Podalgebra, ki jo ta množica generira je najmanjša podalgebra, ki jo $A$ premore, namreč $\mathbbm{2}$. Če je $E$ sam po sebi podalgebra pa generira samega sebe.
\end{primer}

\begin{definicija}
    Podalgebra Boolove algebre A je končno generirana, če je generirana s kakšno končno podmnožico $A$.
\end{definicija}

Naj bo $A$ Boolova algebra in $B$ podalgebra.
Izkaže se, da se poljubni supremumi in infimumi obnašajo nepohlevno.
Supremumi in infimumi lahko spreminjajo vrednosti, lahko jih celo zgubimo ali pa dobimo,
ko prehajamo med $A$ in $B$.

\begin{definicija}
    Naj bo $A$ kompletna Boolova algebra in $B$ njena podalgebra.
    Če za vsako podmnožico $B$ njen supremum (v $A$) leži v $B$ pravimo, da je 
    $B$ kompletna Boolova podalgebra. Seveda to avtomatično implicira identično trditev za infimume.
\end{definicija}

Ta definicija je močnejša, kot če bi zahtevali, da je $B$ sama kompletna boolova algebra.


%%%%%%%%%%%%%%%%%%%%%%%%%%%%%%%%%%%%%%%%%%%%%%%%%%%%%%%%%%%%%%%%%%%%%%%%%%%%%%%%%%%%%%%%%%%%%%%%%%%%%%%%%%%%%%%%%%%%%%%%%%%%%%%%
\section{Homomorfizmi}

\begin{definicija}
    Boolov homomorfizem je taka preslikava $f$ iz Boolova algebre $B$ v 
    Boolovo algebro $A$, da je 
    \begin{align*}
        f(p \wedge q) &= f(p) \wedge f(q),\\
        f(p \vee q) &= f(p) \vee f(q),\\
        f(p') &= (f(p))',
    \end{align*}\footnote{Pišemo tudi \(f(p)'\).}
    za vsaka \(p, q \in B.\)
\end{definicija}

Z lahkoto se prepričamo, da velja \(f(0) = 0\) in \(f(1) = 1\). Posledica tega dejstva je, da ne obstaja trivialni homomorfizem med dvema
neizrojenima Boolovima algebrama. Prepričamo se lahko tudi, da je \(f_{*}(B)\) podalgebra v \(A\).

\begin{definicija}
    Izomorfizem Boolovih algeber je bijekcija, ki je hkrati homomorfizem.
\end{definicija}

Izomorfizem ohranja vse morebitne supremume in infimume, homomorfizem pa 
v splošnem ne. Homomorfizem imenujemo kompleten, če ohranja vse supremume (in posledično infimume), ki obstajajo.

\begin{lema}
    Boolov monomorfizem $f$ iz $B$ v $A$ je kompleten natanko tedaj, ko 
    je slika \(f_{*}(B)\) regularna podalgebra v $A$.
\end{lema}

\begin{proof}
    
\end{proof}

%%%%%%%%%%%%%%%%%%%%%%%%%%%%%%%%%%%%%%%%%%%%%%%%%%%%%%%%%%%%%%%%%%%%%%%%%%%%%%%%%%%%%%%%%%%%%%%%%%%%%%%%%%%%%%%%%%%%%%%%%%%%%%%%
\section{Razširitve homomorfizmov}

\begin{definicija}
    Boolov homomorfizem $f$ je razširitev Boolovega homomorfizma $g$, če je 
    domena $g$ podalgebra domene $f$ in se homomorfizma ujemata na elementih iz domene $g$.
\end{definicija}


\begin{trditev}
    Če se dva homomorfizma ujemata na množici generatorjev domene, tedaj se ujemata povsod na domeni.
\end{trditev}

\begin{proof}
    Naj bosta \(f,g : B \to A\) homomorfizma, ki se ujemata na množici generatorjev $E$.
    \(C := \{p \in B~|~ f(p) = g (p)\}\). Množica $E$ je očitno vsebovana v
    $C$, hkrati pa iz \(p,q \in C\) in 
    \[f(p \vee q) = f(p) \vee f(q) = g(p) \vee g(q) = g(p\vee q)\]
    sledi, da so \(p \vee q,~ p \wedge q\) in \(p'\) tudi elementi $C$.
    Sklepamo, da je $C$ podalgebra v $B$, ki vsebuje $E$, iz česar pa takoj sledi, 
    enakost \(C = B\).
    \label{enolicno}
\end{proof}

%%%%%%%%%%%%%%%%%%%%%%%%%%%%%%%%%%%%%%%%%%%%%%%%%%%%%%%%%%%%%%%%%%%%%%%%%%%%%%%%%%%%%%%%%%%%%%%%%%%%%%%%%%%%%%%%%%%%%%%%%%%%%%%%
\section{Atomi}

\begin{definicija}[Podelement]
    Naj bo $p_0$ element Boolove algebre. Podelement elementa $p_0$ je vsak element
    \(p\), za katerega velja \(p \leq p_0\) oz. ekvivalentno je podelement elementa $p_0$ vsak element oblike
    \(p_0 \wedge p\) za nek element \(p\).
\end{definicija}

\begin{definicija}[Atom]
    {\bf Atom} Boolove algebre je neničeln element, ki nima netrivialnih podelementov oz.\ ko sta njegova edina podelementa natanko $0$ in on sam.
\end{definicija}

\begin{lema}
    Naslednje trditve o elementu $q$ so ekvivalentne:
    \begin{enumerate}
        \item $q$ je atom;
        \item za vsak element $p$ velja natanko ena izmed \(q \leq p\) ali \(q \wedge p = 0\);
        \item za vsak element $p$ velja natanko ena izmed \(q \leq p\) ali \(q \leq p'\);
        \item \(q \neq 0\) in če je $q$ pod \(p \vee r\), potem je \(q \leq p\) ali \(q \leq r\);
        \item \(q \neq 0\) in če je $q$ pod supremumom neke družine \(\{p_i\}\), potem je \(q\) pod \(p_i\) za nek \(i\).
    \end{enumerate}
\end{lema}

\begin{proof} 
    \(1 \Rightarrow 2\)) \(q \wedge p \leq q\), kar implicira ali
    \(q \wedge p = 0\) ali \(q \wedge p = q\) (\(q \leq p\)). Če bi veljalo oboje, bi bil
    $q = 0$, kar pa je protislovje. 

    \(2 \Rightarrow 3\)) Če je \(q \wedge p = 0\) je 
    \[q = q \wedge 1 = q \wedge \left(p \vee p'\right) = \left(q \wedge p\right)
    \vee \left(q \wedge p'\right) = 0 \vee \left(q \wedge p'\right) = 
    \left(q \wedge p'\right),\]
    torej \(q \leq p'\). Če bi veljalo oboje, bi veljalo \(q \leq 0\), kar pa je možno le če 
    je $q = 0$ in spet dobimo protislovje, saj veljata obe izmed \(q \leq p\) ali \(q \wedge p = 0\).

    \(3 \Rightarrow 4\)) Če je \(q = 0\) dobimo protislovje kot prej. Če ne, je \(q \leq p', r'\), potem je 
    \(q = q \wedge (p' \wedge r') = q \wedge (p \vee r)'\) in \(q \leq (p \vee r)'\).

    \(4 \Rightarrow 5\))

\end{proof}

\begin{lema}
    Če je element \(p\) supremum množice atomov \(E\), potem je \(E\) množica vseh atomov pod \(p\).
\end{lema}

\begin{proof}
    TBD
\end{proof}


\begin{definicija}
    Boolova algebra je atomska, če vsak neničeln element dominira vsaj en atom. 
    Boolova algebra je brezatomska, če nima atomov.
\end{definicija}

\begin{lema}
    Naslednje trditve o Boolovi algebri \(A\) so ekvivalentne.
    \begin{enumerate}
        \item A je atomska.
        \item Vsak element je supremum atomov, ki jih dominira.
        \item Enota je supremum množice vseh atomov.
    \end{enumerate}
\end{lema}

\begin{proof}
    TBD
\end{proof}

% \begin{izrek}

% \end{izrek}




%%%%%%%%%%%%%%%%%%%%%%%%%%%%%%%%%%%%%%%%%%%%%%%%%%%%%%%%%%%%%%%%%%%%%%%%%%%%%%%%%%%%%%%%%%%%%%%%%%%%%%%%%%%%%%%%%%%%%%%%%%%%%%%%
\section{Končne Boolove algebre}

\begin{trditev}
    Končna Boolova algebra je atomska.
\end{trditev}

\begin{proof}
    Naj bo \(p\) element končne Boolove algebre. \(p\) je atom, ali pa obstaja 
    neničeln element \(p_1\), ki je strogo manjši. \(p_1\) je atom (in smo končali), ali pa obstaja 
    neničeln element \(p_2\), ki je strogo manjši in tako naprej. Ker je elementov 
    le končno nonogo se mora ta proces ustaviti
    in tako dobimo atom, ki je manjši od \(p\).
\end{proof}

\begin{posledica}
    Vsaka končna Boolova algebra $A$ je izomorfna $\mathcal{P}(n)$, kjer je $n$ število 
    atomov v $A$.
\end{posledica}

\begin{proof}
    Končna Boolova algebra je atomska in kompletna. Naj bo $n$ število atomov v njej.
    $\mathcal{P}(n)$ je tudi atomska kompletna algebra, ki ima $n$ atomov, ki so 
    ravno vse enoelementne množice. Taki pa sta po posledici \ref{resime} izomorfni.
\end{proof}

Posledica nam pove, da ni končnih Boolovih algeber moči, ki ni potenca števila 2.

\begin{posledica}
    Končni Boolovi algebri z enakim številom elementov sta izomorfni.
\end{posledica}

\begin{proof}
    Denimo da imata končni Boolovi algebri $A$ in $B$ enako število elementov.
    Obe sta atomski in imata $m$ in $n$ atomov. Po prejšnji lemi je 
    \(A \cong \mathcal{P}(n)\) in \(B \cong \mathcal{P}(m)\). Ker imata enako
    število elementov velja $n$ = $m$. Ker sta izomorfni isti Boolovi
    algebri sta tudi med samo izomorfni.
\end{proof}

\begin{izrek}
    Boolova identiteta velja v $\mathbbm{2}$ natanko tedaj, 
    ko jo lahko izpeljemo iz aksiomov iz definicije.
\end{izrek}

\begin{trditev}
    Enak nabor identitet velja v vsaki neizrojeni Boolovi algebri.
\end{trditev}

\begin{proof}
    Naj bo $A$ poljubna neizrojena Boolova algebra.
    Identiteta je univerzalna trditev o elementih in operacijah Boolove
    algebre: če valja v $A$, velja tudi v vsaki podalgebri, saj so elementi 
    podalgebre med elementi iz $A$ in so operacije le zožitve operacij v $A$.
    Ker je $A$ neizrojena vsebuje kopijo $\mathbbm{2}$ kot podalgebro, 
    torej lastnosti v $A$ veljajo tudi v $\mathbbm{2}$. Če lastnost velja
    v  $\mathbbm{2}$, se jo da po prejšnjem izreku izpeljati iz aksiomov,
    torej velja tudi v $A$.
\end{proof}

%%%%%%%%%%%%%%%%%%%%%%%%%%%%%%%%%%%%%%%%%%%%%%%%%%%%%%%%%%%%%%%%%%%%%%%%%%%%%%%%%%%%%%%%%%%%%%%%%%%%%%%%%%%%%%%%%%%%%%%%%%%%%%%%
\section{Proste Boolove algebre}

\begin{definicija}
    Množica $E$ generatorjev Boolove algebre $B$ je prosta, če lahko vsako funkcijo iz $E$ v poljubno
    Boolovo algebro $A$ razširimo do homomorfizma iz $B$ v $A$. Tedaj pravimo, da $E$ prosto generira $B$ oz.
    $B$ je prosta na $E$. Boolova algebra je prosta, če premore prosto množico generatorjev.
\end{definicija}

Definicijo lahko razložimo s pomočjo naslednjega diagrama. $h$ je identična preslikava iz $E$ v $B$,
$g$ je preslikava, definirana na $E$, $f$ pa je porojen homomorfizem
iz definicije.

\begin{equation*}
  \xymatrix{
    {E}
    \ar[r]^{h}
    \ar[dr]_{g}
    &
    {B}
    \ar@{-->}[d]^f
    \\
    &
    {A}
  }
\end{equation*}

Posledica trditve \ref{enolicno} je, da je dobljeni homomorfizem enolično
določen z $g$.

Če imamo dve Boolovi algebri $B_1$ in $B_2$, ki sta prosti generirani
z ekvipolentnima množicama generatorjev $E_1$ in $E_2$,
sta ti izomorfni. Izomorfizem 
je porojen z vsako bijekcijo $g$ med $E_1$ in $E_2$.
Predpostavka o prostosti zagotovi homomorfizma \(f_1: B_1 \to B_2\) in \(f_2: B_2 \to B_1\),
ki razširita \(g\) in \(g^{-1}\). \(f_1 \circ f_2\) je endomorfizem
$B_1$, ki razširi \(g^{-1} \circ g\), torej identiteto na $E_1$. Identiteta na $B_1$ sama
pa je že razširitev te preslikave, torej je zaradi enoličnosti
\(f_1 \circ f_2 = id_{B_1}\). Podobno vidimo, da je \(f_2 \circ f_1 = id_{B_2}\).
Torej je \(f_1\) izomorfizem, \(f_2\) pa njegov inverz.



\begin{table*}[!htb]
    \begin{minipage}{.5\linewidth}
      \centering
        \begin{equation*}
            \xymatrix{
            {E_1}
            \ar[r]^{h_1}
            \ar@{->}[d]_{g}
            &
            {B_1}
            \ar@<0.5ex>@{-->}[d]^{f_1}
            \\
            {E_2}
            \ar[r]_{h_2}
            &
            {B_2}   
            \ar@<0.5ex>@{-->}[u]^{f_2}
            }
        \end{equation*}
    \end{minipage}%
    \begin{minipage}{.5\linewidth}
      \centering
        \begin{equation*}
            \xymatrix{
              {E_1}
              \ar[r]^{h_1}
              \ar@{->}[dr]_{g^{-1} \circ g}
              &
              {B_1}
              \ar@{-->}[d]^{f_1}
              \\
              &
              {B_1}      
            }
        \end{equation*}
    \end{minipage} 
\end{table*}

\begin{trditev}
    Neskončna prosta Boolova algebra je brezatomska.
\end{trditev}

































\newpage
\section*{Angleško-slovenski slovar strokovnih izrazov}


\geslo{Boolean algebra}{Boolova algebra}

\geslo{Degenerate}{Izrojena}

\geslo{The principle of duality}{Princip dualnosti}

\geslo{Lower (upper) bound}{Spodnja(zgornja) meja}

\geslo{Least upper bound}{Najmanjša zgornja meja}

\geslo{Greatest lower bound}{Največja spodnja meja}

\geslo{Complete Boolean algebra}{Kompletna Boolova algebra}

\geslo{Finite-cofinite Boolean algebra}{Končno-končna Boolova algebra}

\geslo{Countable-cocountable Boolean algebra}{Števno-števna Boolova algebra}

\geslo{Atom}{Atom}

\geslo{}{}

\geslo{}{}

\geslo{}{}

\geslo{}{}

\geslo{}{}

\geslo{}{}

\geslo{}{}

\geslo{}{}

\geslo{}{}


\geslo{}{}






\begin{thebibliography}{1}
\bibitem{GSHP}
Givant, Steven; Halmos, Paul. “Introduction to Boolean Algebras (Undergraduate Texts in Mathematics),” Springer (2009).
\end{thebibliography}

\end{document}
