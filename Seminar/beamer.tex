\documentclass{beamer}

\usepackage[T1]{fontenc}
\usepackage[utf8]{inputenc}
\usepackage[slovene]{babel}

\usepackage{palatino}
\usefonttheme{serif}
\usetheme{Berlin}
\usecolortheme{default}
\useinnertheme[shadows]{rounded}
\useoutertheme{infolines}

\usepackage[T1]{fontenc}
\usepackage[utf8]{inputenc}
\usepackage[slovene]{babel}
\usepackage{textcase}
\usepackage{amsmath,amsfonts,amssymb,amsthm}
\usepackage{xypic}
\usepackage{booktabs}
\usepackage{multirow}
\usepackage{siunitx}
\usepackage{bbm}
\sisetup{output-decimal-marker={,},group-separator={.}}


\newcommand{\RR}{\mathbb{R}}
\newcommand{\NN}{\mathbb{N}}
\newcommand{\PP}{\mathbb{P}}
\newcommand{\ZZ}{\mathbb{Z}}
\newcommand{\geslo}[2]{\noindent\textbf{#1} \quad \hangindent=1cm #2\\[-1pc]}



{\theoremstyle{theorem}
\newtheorem{izrek}{Izrek}[section]
\newtheorem{lema}[izrek]{Lema}
\newtheorem{trditev}[izrek]{Trditev}
\newtheorem{posledica}[izrek]{Posledica}
}
{\theoremstyle{definition}
\newtheorem{definicija}[izrek]{Definicija}
\newtheorem{vaja}[izrek]{Vaja}
\newtheorem{prob}[izrek]{Nemogoč problem}
\newtheorem{opomba}[izrek]{Opomba}
\newtheorem{domneva}[izrek]{Domneva}
\newtheorem{odprt}[izrek]{Odprt problem}
\newtheorem{primer}[izrek]{Primer}
}
\def\qed{$\hfill\Box$}   % konec dokaza
\def\qedm{\qquad\Box}   % konec dokaza v matematičnem načinu

\usepackage{amsmath}
\usepackage{amssymb}

\beamertemplatenavigationsymbolsempty




% \newtheorem{izrek}{izrek}
\newcommand{\N}{\mathbb{N}}
\newcommand{\abs}[1]{|#1|}
\newcommand{\nastevanje}[2]{#1_1 + #1_2 + \ldots + #1_{#2}}

%\bibliographystyle{siam}
%\bibliography{magic}
%\cite[Poglavje 3]{}


\title{Prosta Boolova algebra}
\subtitle{Seminar}
\author{Luka Ponikvar}
\institute [FMF] {Fakulteta za matematiko in fiziko}
\date{}
% ===================================================================
\begin{document}
  

% Naslovna stran
\begin{frame}
 \titlepage
\end{frame}
% -------------------------------------------------------------------

% Prosojnica: Kratek pregled
% \begin{frame}
%   \frametitle{Kratek pregled}
% \tableofcontents[pausesections]
  

% \end{frame}
% ===================================================================

% \section{Osnovna razdelitev vsebine}
%\begin{frame}
  
%\end{frame}
% -------------------------------------------------------------------

% Prosojnica: Naštevanje
  \begin{frame}
    \begin{columns}
        \begin{column}{\textwidth}
            \begin{definicija}[Boolova algebra]
                {\bf Boolova algebra} je neprazna množica \(A\) skupaj z binarnima operacijama 
                \(\vee\)  in \(\wedge\), unarno operacijo \(\neg\)
                 in dvema  
                elementoma \(0\) in \(1\), ki skupaj zadoščajo sledečim aksiomom:
                \begin{align}
                    \label{eq0}
                    \neg 0 &= 1, & \neg 1 =& 0,\\ \label{eq1}
                    p \wedge 0 &= 0, & p \vee 1 =& 1,\\ \label{eq2}
                    p \wedge 1 &= p, & p \vee 0 =& p,\\ \label{eq3}
                    p \wedge \neg p &= 0, & p \vee \neg p =& 1,\\ \label{eq4}
                        \neg (\neg p) &= p,\\ \label{eq5}
                    p \wedge p &= p, & p \vee p =& p,\\ \label{eq6}
                    \neg (p \wedge q) &= \neg p \vee \neg q, & \neg (p \vee q) =& \neg p \wedge \neg q,\\ \label{eq7}
                    p \wedge q &= q \wedge p, & p \vee q =& q \vee p,\\ \label{eq8}
                    p \wedge (q \wedge r) &= (p \wedge q) \wedge r, & p \wedge (q \vee r) =& (p \vee q) \vee r,\\ \label{eq9}
                    p \wedge (q \vee r) &= (p \wedge q) \vee (p \wedge r), & p \vee (q \wedge r) =& (p \vee q) \wedge (p \vee r).
                \end{align}
            \end{definicija}
        \end{column}
    \end{columns}
  \end{frame}
% ===================================================================

% Prosojnica: Stolpci
\begin{frame}
    \begin{definicija}[Boolov polinom]
        {\bf Boolov polinom} je izraz, sestavljen iz konstant \(0\) in \(1\), neznank \(p_0\), \(\ldots\) , \(p_n\), s pomočjo standardnih operacij
        meet, join in komplementa.
    \end{definicija}
\pause
    \begin{definicija}[Boolova Podalgebra]
        Boolova podalgebra Boolove algebre $A$ je neprazna podmnožica $B$ množice $A$, ki je z zožitvijo 
        operacij Boolova algebra.
    \end{definicija}
\pause
    \begin{definicija}[Boolov homomorfizem]
        Boolov homomorfizem je taka preslikava $f$ iz Boolova algebre $B$ v 
        Boolovo algebro $A$, da je 
        \begin{align*}
            f(p \wedge q) &= f(p) \wedge f(q),\\
            f(p \vee q) &= f(p) \vee f(q),\\
            f(p') &= (f(p))',
        \end{align*}
        za vsaka \(p, q \in B.\)
    \end{definicija}
\end{frame}
  
\begin{frame}
    \begin{trditev}
        Če se dva homomorfizma ujemata na množici generatorjev domene, tedaj se ujemata povsod na domeni.
    \end{trditev}
\pause
    \begin{proof}
        Naj bosta \(f,g : B \to A\) homomorfizma, ki se ujemata na množici generatorjev $E$.
        \(C := \{p \in B~|~ f(p) = g (p)\}\). Množica $E$ je očitno vsebovana v
        $C$, hkrati pa iz \(p,q \in C\) in 
        \[f(p \vee q) = f(p) \vee f(q) = g(p) \vee g(q) = g(p\vee q)\]
        sledi, da so \(p \vee q,~ p \wedge q\) in \(p'\) tudi elementi $C$.
        Sklepamo, da je $C$ podalgebra v $B$, ki vsebuje $E$, iz česar pa takoj sledi, 
        enakost \(C = B\).
    \end{proof}
\end{frame}

\begin{frame}
    \begin{definicija}[Prosta Boolova algebra]
        Množica $E$ generatorjev Boolove algebre $B$ je prosta, če lahko vsako funkcijo iz $E$ v poljubno
        Boolovo algebro $A$ razširimo do homomorfizma iz $B$ v $A$. Tedaj pravimo, da $E$ prosto generira $B$ oz.
        $B$ je prosta na $E$. Boolova algebra je prosta, če premore prosto množico generatorjev.
    \end{definicija}
\pause
    \begin{equation*}
        \xymatrix{
          {E}
          \ar[r]^{h}
          \ar[dr]_{g}
          &
          {B}
          \ar@{-->}[d]^f
          \\
          &
          {A}
        }
      \end{equation*}
\end{frame}

\begin{frame}
    \begin{table}
        \begin{minipage}{.5\linewidth}
          \centering
            \begin{equation*}
                \xymatrix{
                {E_1}
                \ar[r]^{h_1}
                \ar@{->}[d]_{g}
                &
                {B_1}
                \ar@<0.5ex>@{-->}[d]^{f_1}
                \\
                {E_2}
                \ar[r]_{h_2}
                &
                {B_2}   
                \ar@<0.5ex>@{-->}[u]^{f_2}
                }
            \end{equation*}
        \end{minipage}
        \begin{minipage}{.5\linewidth}
          \centering
            \begin{equation*}
                \xymatrix{
                  {E_1}
                  \ar[r]^{h_1}
                  \ar@{->}[dr]_{g^{-1} \circ g}
                  &
                  {B_1}
                  \ar@{-->}[d]^{f_1}
                  \\
                  &
                  {B_1}      
                }
            \end{equation*}
        \end{minipage} 
    \end{table}
\end{frame}

  \end{document}