\documentclass{amsart}
\usepackage[T1]{fontenc}
\usepackage[utf8]{inputenc}
\usepackage[slovene]{babel}
\usepackage{textcase}
\usepackage{amsmath,amsfonts,amssymb,amsthm}
\usepackage{xypic}
\usepackage{booktabs}
\usepackage{multirow}
\usepackage{siunitx}
\sisetup{output-decimal-marker={,},group-separator={.}}


\newcommand{\RR}{\mathbb{R}}
\newcommand{\NN}{\mathbb{N}}
\newcommand{\PP}{\mathbb{P}}
\newcommand{\ZZ}{\mathbb{Z}}
\newcommand{\geslo}[2]{\noindent\textbf{#1} \quad \hangindent=1cm #2\\[-1pc]}



{\theoremstyle{theorem}
\newtheorem{izrek}{Izrek}[section]
\newtheorem{lema}[izrek]{Lema}
\newtheorem{trditev}[izrek]{Trditev}
\newtheorem{posledica}[izrek]{Posledica}
}
{\theoremstyle{definition}
\newtheorem{definicija}[izrek]{Definicija}
\newtheorem{vaja}[izrek]{Vaja}
\newtheorem{prob}[izrek]{Nemogoč problem}
\newtheorem{opomba}[izrek]{Opomba}
\newtheorem{domneva}[izrek]{Domneva}
\newtheorem{odprt}[izrek]{Odprt problem}
\newtheorem{primer}[izrek]{Primer}
}
\def\qed{$\hfill\Box$}   % konec dokaza
\def\qedm{\qquad\Box}   % konec dokaza v matematičnem načinu


% \title[kratek naslov]{poln naslov}
\title[Prosta Boolova algebra]{Prosta Boolova algebra\\\Large Seminar}
% Podatki o avtorjih
\author{Luka Ponikvar}
\address{Luka Ponikvar\\
Fakulteta za matematiko in fiziko\\
Oddelek za matematiko\\
Jadranska 21\\
1000 Ljubljana\\
Slovenija}
\email{lp29353@student.uni-lj.si}

\begin{document}

\maketitle


\begin{abstract}
    Pričetek toerije Boolovih algeber je nepresenetljivo pripisan Geogu Boolu,
    ki 
\end{abstract}


%%%%%%%%%%%%%%%%%%%%%%%%%%%%%%%%%%%%%%%%%%%%%%%%%%%%%%%%%%%%%%%%%%%%%%%%%%%%%%%%%%%%%%%%%%%%%%%%%%%%%%%%%%%%%%%%%%%%%%%%%%%%%%%%
\section{Boolove algebre}

\begin{definicija}[Boolova algebra]
    {\bf Boolova algebra} je neprazna množica \(A\), skupaj z binarnima operacijama \(\vee\) \footnote{Imenujemo jo ``meet''\ oz.\ 
    ``ali''.} in \(\wedge\)\footnote{Imenujemo jo ``join''\ oz.\ ``in''.}, unarno operacijo \(\neg\)
    \footnote{Imenujemo jo negacija, označujemo pa tudi kot \('\).} in dvema odlikovanima 
    elementoma \(0\) in \(1\), ki skupaj zadoščajo sledečim aksiomom\footnote{Negacija ima najvišjo prioriteto, medtem ko ima ``in'' višjo prioriteto kot ``ali''.}:

    \begin{align}
        \neg 0 &= 1, & \neg 1 =& 0,\\ \label{eq1}
        p \wedge 0 &= 0, & p \vee 1 =& 1,\\ \label{eq2}
        p \wedge 1 &= p, & p \vee 0 =& p,\\ \label{eq3}
        p \wedge \neg p &= 0, & p \vee \neg p =& 1,\\ \label{eq4}
         \neg (\neg p) &= p,\\ \label{eq5}
        p \wedge p &= p, & p \vee p =& p,\\ \label{eq6}
        \neg (p \wedge q) &= \neg p \vee \neg q, & \neg (p \vee q) =& \neg p \wedge \neg q,\\ \label{eq7}
        p \wedge q &= q \wedge p, & p \vee q =& q \vee p,\\ \label{eq8}
        p \wedge (q \wedge r) &= (p \wedge q) \wedge r, & p \wedge (q \vee r) =& (p \vee q) \vee r,\\ \label{eq9}
        p \wedge (q \vee r) &= (p \wedge q) \vee (p \wedge r), & p \vee (q \wedge r) =& (p \vee q) \wedge (p \vee r). \label{eq10}
    \end{align}
    
\end{definicija}

\begin{primer}[Izrojena Boolova algebra]
    Najenostavnejši je primer izrojene Boolove algebre, ki je potenčna množica prazne množice:
    \begin{align*}
        \mathcal{P}\left(\emptyset\right) = \{\emptyset\}.
    \end{align*}
    Operacije na tej množici definiramo kot konstantne preslikave, ki vse slikajo v \[0 = 1 = \emptyset.\]
\end{primer}

\begin{primer}[Boolova algebra z dvema elementoma]
    Najmanjši primer neizrojene Boolove algebre je potenčna množica enojca\footnote{Element množice smo označili kar z \(\infty\).}:
    \begin{align*}
        \mathcal{P}\left(\{\infty\}\right) = \{\emptyset, \{\infty\}\}.
    \end{align*}
    Taka Boolova algebra ima le dva elementa:
    \begin{align*}
        \emptyset = 0, && \{\infty\} = 1.
    \end{align*}
    Operaciji join in meet sta predstavljeni z naslednjima tabelama:
    \begin{align*}
    \begin{tabular}{c|c|c|}
        \(\vee\) & $0$ & $1$ \\
        \hline
        $0$ & \(0\) &$1$ \\
        \hline
        $1$ & \(1\) & $1$ \\
        \hline
    \end{tabular}&& \text{in} &&
    \begin{tabular}{c|c|c|}
        \(\wedge\) & $0$ & $1$ \\
        \hline
        $0$ & $0$ &$0$ \\
        \hline
        $1$ & $0$ & $1$ \\
        \hline
    \end{tabular},
\end{align*}
komplementacija, pa $0$ preslika v $1$ in obratno.
\end{primer}

%%%%%%%%%%%%%%%%%%%%%%%%%%%%%%%%%%%%%%%%%%%%%%%%%%%%%%%%%%%%%%%%%%%%%%%%%%%%%%%%%%%%%%%%%%%%%%%%%%%%%%%%%%%%%%%%%%%%%%%%%%%%%%%%
\section{Princip dualnosti}

\begin{definicija}[Boolov polinom]
    {\bf Boolov polinom} je izraz, sestavljen iz konstant \(0\) in \(1\), neznank \(p_0\), \(\ldots\) , \(p_n\), s pomočjo standardnih operacij
    meet, join in komplementa.
\end{definicija}

\begin{primer}[Boolov polinom]
    Primer polinoma je
    \[p \wedge (q \vee r).\]
    Ta polinom, pa je na pogled zelo podoben polinomu
    \[p \vee (q \wedge r),\]
    kar motivira naslednjo definicijo.
\end{primer}

\begin{definicija}[Dualnost]
    Naj bo \(f(p_1, \ldots ,p_n)\) boolov polinom v n spremenljivkah. Takemu polinomu lahko priredimo
    tri nove polinome:
    \begin{enumerate}
        \item {\bf komplement polinoma} \(f(p_1, \ldots ,p_n)\):
            \[f'(p_1, \ldots ,p_n),\]
        \item {\bf dual polinoma} \(f(p_1, \ldots ,p_n)\):
            \[f'(p_1', \ldots ,p_n'),\]
        \item {\bf kontradual polinoma} \(f(p_1, \ldots ,p_n)\):
            \[f(p_1', \ldots ,p'_n).\]
    \end{enumerate}
\end{definicija}
\begin{opomba}
    V resnici obstaja grupa \(G\), ki deluje na množici \(\mathcal{BP}\) vseh boolovih polinomov.
    Obstajajo štiri funkcije, ki bijektivno preslikajo množico \(\mathcal{BP}\) nazaj nase:
    \begin{enumerate}
        \item Identična funkcija:
        \[id: \mathcal{BP} \rightarrow \mathcal{BP}\]
        \[id: f(p_1, \ldots ,p_n) \mapsto f(p_1, \ldots ,p_n).\]
        \item Komplementna funkcija:
        \[c: \mathcal{BP} \rightarrow \mathcal{BP}\]
        \[c: f(p_1, \ldots ,p_n) \mapsto f'(p_1, \ldots ,p_n).\]
        \item Dualna funkcija:
        \[d: \mathcal{BP} \rightarrow \mathcal{BP}\]
        \[d: f(p_1, \ldots ,p_n) \mapsto f'(p_1', \ldots ,p_n').\]
        \item Kontradualna funkcija:
        \[k: \mathcal{BP} \rightarrow \mathcal{BP}\]
        \[k: f(p_1, \ldots ,p_n) \mapsto f(p_1', \ldots ,p_n').\]
    \end{enumerate}

    Grupa \(G\) je zaprta za operacijo \(\circ\):
    \begin{table*}[htp]
        \begin{tabular}{c|c|c|c|c|}
            \(\circ\) & $id$ & $c$ & $d$ & $k$ \\
            \hline
            $id$ & $id$ &$c$ & $d$ & $k$ \\
            \hline
            $c$ & $c$ & $id$ & $k$ & $d$ \\
            \hline
            $d$ & $d$ & $k$ & $id$ & $c$ \\
            \hline
            $k$ & $k$ & $d$ & $c$ & $id$ \\
            \hline
        \end{tabular}.
    \end{table*}\\
    Opazimo, da je \(G \cong \ZZ_2  \oplus \ZZ_2 \), torej je Kleinova četverka.
\end{opomba}

\begin{opomba}
    Praktična posledica principa dualnosti je moč dokazati le polovico izrekov in trditev, 
    saj druga polovica sledi iz tega principa.
\end{opomba}

%%%%%%%%%%%%%%%%%%%%%%%%%%%%%%%%%%%%%%%%%%%%%%%%%%%%%%%%%%%%%%%%%%%%%%%%%%%%%%%%%%%%%%%%%%%%%%%%%%%%%%%%%%%%%%%%%%%%%%%%%%%%%%%%
\section{Urejenost}

V tem razdelku delujemo v poljubni Boolovi algebri A.

\begin{lema}
    \(\left(p \vee q\right) \wedge p = p\) in \(\left(p \wedge q\right) \vee p = p\).
\end{lema}

\begin{proof} 
    \begin{align*}
        \left(p \vee q\right) \wedge p &\stackrel{\ref{eq3}}{=} \left(p \vee q\right) \wedge \left(p \vee 0 \right) \\
                    &\stackrel{}{=} p \vee \left(q \wedge 0\right)\\
                    &\stackrel{}{=} p \vee 0\\
                    &\stackrel{}{=} p 
    \end{align*}
    Druga furmula sledi iz dualnosti.

\end{proof}


\begin{lema}
    \(p \wedge q = p\) natanko tedaj ko \(p \vee q = q\).
\end{lema}

\begin{proof}
    Če je \(p \wedge q = p\), je 
    \begin{align*}
        p \vee q &\stackrel{}{=} \left(p \wedge q\right) \vee q \\
                    &\stackrel{}{=} \left(p \vee q\right) \wedge \left(q \vee q\right) \\
                    &\stackrel{}{=} \left(p \vee q\right) \wedge q\\
                    &\stackrel{}{=} q 
    \end{align*}
\end{proof}


\newpage
\section*{Angleško-slovenski slovar strokovnih izrazov}


\geslo{Boolean algebra}{Boolova algebra}

\geslo{Degenerate}{Izrojena}

\geslo{The principle of duality}{Princip dualnosti}






\begin{thebibliography}{1}
\bibitem{AiZ}
Givant, Steven; Halmos, Paul. “IntroDuction to Boolean Algebras (Undergraduate Texts in Mathematics),” Springer (2009).\bibitem{CaW}
N.~Calkin in H.~S.~Wilf, Recounting the rationals,
\emph{Amer.~Math.~Monthly}  \textbf{107}  (2000),  360--363.
\bibitem{Gra}
J.~Grasselli, \emph{Elementarna teorija števil}, DMFA -- založništvo, Ljubljana, 2009.
\end{thebibliography}

\end{document}
